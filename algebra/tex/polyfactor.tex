\begin{section}{Factorization of Polynomials Over a Field}

The following theorem is a fundamental tool for working with polynomials, it is known as the {\bf Division Algorithm for $F[x]$} since it is analogous to the Division Algorithm for $\mathbb{Z}$.

\begin{thm}
Let $f(x),g(x) \in F[x]$ be polynomials of degree $n$ and $m$ with leading terms $a_{n}x^{n}, b_{m}x^{m}$ respectively, with $a_{n}, b_{m} \neq 0$ and $m > 0$. Then there exist unique polynomials $q(x),r(x) \in F[x]$ such that $$f(x) = g(x)q(x) + r(x),$$ where either $r(x) = 0$ or $\textrm{Deg}(r(x)) < m$.
\end{thm} 

The following corollary follows immediately from the division algorithm for $F[x]$, it is known also as the {\bf Factor Theorem}. 

\begin{cor}
An element $a \in F$ is a zero of $f(x) \in F[x]$ if and only if $(x-a) \in F[x]$ is a factor of $f(x)$.
\end{cor}

\begin{cor}
A nonzero polynomial $f(x) \in F[x]$ with $\textrm{Deg}(f(x)) = n$ can have at most $n$ zeros in $F$.
\end{cor}

\begin{cor}
If $G$ is a finite subgroup of the multiplicative group $(F/\{0\},\cdot)$ of a field $F$, then $G$ is cyclic. That is, the multiplicative group of all nonzero elements of a finite field is cyclic.
\end{cor}

\begin{defn}
A nonconstant polynomial $f(x) \in F[x]$ is an {\bf irreducible polynomial} over $F$ if $f(x)$ cannot be expressed as a product $g(x)h(x)$ for polynomials $g(x),h(x) \in F[x]$ of degree $\textrm{Deg}(g(x)) \textrm{ and } \textrm{Deg}(h(x)) < \textrm{Deg}(f(x))$. If $f(x)$ is not irreducible over $F$, then it is a {\bf reducible polynomial} over $F$.
\end{defn}

\begin{thm}
Let $f(x) \in F[x]$ be of degree $2$ or $3$. Then $f(x)$ is reducible over $F$ if and only if it has a zero in $F$.
\end{thm}

\begin{thm}
If $f(x) \in \mathbb{Z}[x]$, then $f(x)$ factors into a product $r(x)s(x)$ of two polynomials of lower degree where $r(x),s(x) \in \mathbb{Q}[x]$ if and only if $f(x)$ has such a factorization with polynomials in $\mathbb{Z}[x]$ of the same degrees as $r(x),s(x)$. 
\end{thm}

\begin{cor}
If $f(x) = x^{n} + a_{n-1}x^{n-1} + \dots + a_{0} \in \mathbb{Z}[x]$ with $a_{0} \neq 0$, and if $f(x)$ has a zero in $\mathbb{Q}$ then it has a zero $m \in \mathbb{Z}$ such that $m$ divides $a_{0}$.
\end{cor}

The following theorem can be incredibly useful for determining the reducibility of polynomials over $\mathbb{Q}$, it is known as the {\bf Eisenstien Criterion}.

\begin{thm}
Let $p \in \mathbb{Z}$ be prime and suppose that $f(x) = a_{n}x^{n} + \dots + a_{0} \in \mathbb{Z}[x]$ and $a_{n} \neq 0(\textrm{mod}p)$, but $a_{i} \equiv 0(\textrm{mod}p), \forall i<n,$ with $a_{0} \neq 0(\textrm{mod}p^{2})$. Then $f(x)$ is irreducible over $\mathbb{Q}$.
\end{thm}

\begin{cor}
The polynomial $$\phi_{p}(x) = \frac{x^{p}-1}{x-1} = x^{p-1} + x^{p-1} + \dots + x + 1$$ is irreducible over $\mathbb{Q}$ for any prime $p$. This polynomial is known as the {\bf $p^{\textrm{th}}$ cyclotomic polynomial.}
\end{cor}

\begin{thm}
Every nonconstant polynomial $f(x) \in F[x]$ can be factored in $F[x]$ into a product of unique irreducible polynomials (up to reordering and choice of a nonzero constant $a_{0} \in F$).
\end{thm}

\end{section}