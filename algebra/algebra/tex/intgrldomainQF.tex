\begin{section}{Fields of Quotients}

We review the construction of a field $F$ from an integral domain $D$.

\begin{rmk}
Our motivating example here is the construction of the field $\mathbb{Q}$ from the integral domain $\mathbb{Z}$, since we have already observed that the only units in $\mathbb{Z}$ are $\{-1,1\}$ and therefore $\mathbb{Z}$ does not have a field structure. This is the notion of an integral domain``embedded'' in a field. 
\end{rmk}

\begin{prop}
The field $F$ can be constructed from an integral domain $D$ according to the following process.
\begin{enumerate}[(1)]
\item Determine the elements of $F$.
\begin{enumerate}[(a)] % BEGIN STEP 1
\item We take the direct product of $D$ with itself to get the set of quotients
$$D \times D = \{(a,b) \colon a,b \in D\}.$$
\item To avoid division by zero\footnote{By analogy to $\mathbb{Z} \hookrightarrow \mathbb{Q}$, observe that the pair $(n,m)$ represents the ``formal quotient'' $n/m$. Then we want to exclude ordered pairs of the form $(a,0)$.} we take $$S = D \times D/\{0\} = \{(a,b) \colon a,b \in D \textrm{ and } b \neq 0\}.$$
\item Now mod out\footnote{Continuing with the analogy to $\mathbb{Z} \hookrightarrow \mathbb{Q}$, note also that $(2,3) = (4,6)$.} by the equivalence\footnote{It is non-trivial to prove that this relation is transitive!} relation $(a,b) \sim (c,d) \iff ad = bc.$
\item Then define the elements of $F$ to be the set of all equivalence classes $[(a,b)]$ for $(a,b) \in S$.
\end{enumerate} % END STEP 1
\item Define addition and multiplication for $F$.
\begin{enumerate}[(a)] % BEGIN STEP 2
\item Addition is well-defined for $[(a,b)]$ and $[(c,d)]$ in $F$, where
$$[(a,b)] + [(c,d)] = [(ad + bc, bd)].$$
\item Multiplication is well-defined for $[(a,b)]$ and $[(c,d)]$ in $F$, where
$$[(a,b)][(c,d)] = [(ac, bd)].$$
\item Note that $bd \neq 0$ since $D$ contains no divisors of zero by definition of an integral domain.
\end{enumerate} %END STEP 2
\item Check that the field axioms hold for $F$ under these operations.
\begin{enumerate}[(a)] % BEGIN STEP 3
\item Check that $F$ is an Abelian group under addition with additive identity $[(0,1)]$ and where $[(-a,b)]$ is the additive inverse for $[(a,b)]$.
\item Check that the non-zero elements of $F$ form an Abelian group under multiplication, where $[(1,1)]$ is the muliplicative identity and $[(b,a)]$ is the multiplicative inverse for $[(a,b)]$ where $a \neq 0$ in $D$.
\item Check that the distributive laws hold in $F$.
\end{enumerate} %END STEP 3
\item Show that $D \hookrightarrow F$ is an integral subdomain of the constructed field $F$.
\begin{enumerate}[(a)] % BEGIN STEP 4
\item Check that the morphism $\iota \colon D \to F$ defined by the function $\iota(a) = [(a,1)]$ is an isomorphism between $D$ and a subring of $F$.
\item Check that $\iota(D)$ is a subdomain of $F$. That is, verify
$$[(a,b)] = [(a,1)][(1,b)] = [(a,1)]/[(b,1)] = \iota(a)/\iota(b).$$
\end{enumerate} %END STEP 4
\end{enumerate}
\end{prop}

\begin{thm}
Let $D$ be an integral domain and $F$ a field, then we can always construct an embedding $D \hookrightarrow F$ such that any element of $F$ can be expressed as a quotient of elements in $D$.
\end{thm}

\begin{prop}
Every field $L$ containing an integral domain $D$ contains a subfield of quotients of $D$.
\end{prop}

\begin{prop}
Any two fields of quotients of an integral domain $D$ are isomorphic to one another.
\end{prop}

\begin{thm}
Let $F$ be a field of quotients of $D$ and let $L \supset D$ be a field. Then there exists an isomorphism $\psi \colon F \to L$ given by the function $\psi(a) = a$ for $a \in D$.
\end{thm}

\end{section}