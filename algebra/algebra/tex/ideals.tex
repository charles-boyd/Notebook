\begin{section}{Prime and Maximal Ideals}

We consider more deeply the structural properties of a quotient ring over an ideal $R/N$.

\begin{danger}
Keep in mind that when we mention a ring $R$, it is implicit $R$ {\em has a unit} unless otherwise noted. Many of the results here do not generalize to semi-rings!
\end{danger}

\begin{prop}
Let $N \subset R$ be an ideal. If $N$ contains a unit of $R$, then $N = R$.
\end{prop}

\begin{prop}
A field contains no proper (nontrivial) ideals.
\end{prop}

\begin{defn}
A {\bf maximal ideal} of a ring $R$ is an ideal $M \subset R$ such that there is no proper ideal $N \subset R$ properly containing $M$.
\end{defn}

\begin{thm}
Let $R$ be a commutative ring. Then $M \subset R$ is a maximal ideal if and only $R/M$ is a field.
\end{thm}

\begin{cor}
A commutative ring is a field if and only if it has no proper (nontrivial) ideals.
\end{cor}

\begin{prop}
If $R$ is a ring which is not even an integral domain, then it is possible for $R/N$ to still be a field for some ideal $N \subset R$.
\end{prop}

\begin{defn}
An ideal $N \subset R$ in a commutative ring is a {\bf prime ideal} if $$ab \in N \implies a \in N \textrm{ or } b \in N \textrm{ for } a,b \in R.$$
\end{defn}

\begin{thm}
Let $R$ be a commutative ring containing a proper ideal $N \subset R$. Then $R/N$ is an integral domain if and only if $N$ is a prime ideal in $R$.
\end{thm}

\begin{cor}
Every maximal ideal in a commutative ring is a prime ideal.
\end{cor}

\begin{thm}
A field $F$ is either of prime characteristic $p$ and contains a subfield isomorphic to $\mathbb{Z}_{p}$, or is of characteristic $0$ and contains a subfield isomorphic to $\mathbb{Q}$.
\end{thm}

\begin{defn}
For a commutative ring $R$ with $a \in R$, the ideal $\<a\> = \{ra \colon r \in R\}$ of all multiples of $a$ is called the {\bf principal ideal generated by} $a$ and any ideal $N \subset R$ where $N = \<a\>$ is called a {\bf principal ideal}.  
\end{defn}

\begin{defn}
If $F$ is a field, then every ideal in $F[x]$ is principal.
\end{defn}

\begin{prop}
A (non-trivial) ideal $\<p(x)\> \subset F[x]$ is maximal if and only if $p(x)$ is irreducible over $F$.
\end{prop}

\begin{ex}
Some examples of prime and maximal ideals.
\begin{enumerate}[(a)]
\item The subset $N = \{0,3\}$ of $\mathbb{Z}_{6}$ is a prime ideal and $\mathbb{Z}_{6}/N \simeq \mathbb{Z}_{3}$.
\item $\{0\}$ is a prime ideal in $\mathbb{Z}$ and actually in any integral domain.
\item $\mathbb{Z}/n\mathbb{Z} \simeq \mathbb{Z}_{n}$ is an integral domain, so $n\mathbb{Z}$ is a prime ideal of $\mathbb{Z}$.
\item $\mathbb{Z}/p\mathbb{Z} \simeq \mathbb{Z}_{p}$ is a field, so $p\mathbb{Z}$ is a maximal ideal of $\mathbb{Z}$.
\item $\mathbb{Z} \times \{0\}$ is a prime ideal of $\mathbb{Z} \times \mathbb{Z}$. Hence, their quotient ring is isomorphic to the integral domain $\mathbb{Z}$.
\item Every ideal of the ring $\mathbb{Z}$ is of the form $n\mathbb{Z} = \<n\>$, so every ideal of $\mathbb{Z}$ is a principal ideal.
\item $p(x) = x^{3} + 3x + 2 \in \mathbb{Z}_{5}[x]$ is irreducible, so $\<p(x)\>$ is maximal and $\mathbb{Z}_{5}[x]/\<p(x)\>$ is a field. 
\item $q(x) = x^{2} - 2 \in \mathbb{Q}[x]$ is irreducible, so $\<q(x)\>$ is maximal and $\mathbb{Q}[x]/\<q(x)\>$ is a field.
\end{enumerate}
\end{ex}

\begin{thm}
Let $p(x)$ be an irreducible polynomial in $F[x]$. If $$p(x) \textrm{ divides } r(x)s(x), \textrm{ for } r(x),s(x) \in F[x],$$ then $p(x)$ either divides $r(x)$ or $s(x)$.
\end{thm}

\end{section}
